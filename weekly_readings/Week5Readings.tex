\documentclass[12pt]{article}

% fonts
\usepackage[scaled=0.92]{helvet}   % set Helvetica as the sans-serif font
\renewcommand{\rmdefault}{ptm}     % set Times as the default text font

% dmb: not mandatory, but i recommend you use mtpro for math fonts.
% there is a free version called mtprolite.

% \usepackage[amssymbols,subscriptcorrection,slantedGreek,nofontinfo]{mtpro2}

\usepackage[T1]{fontenc}
\usepackage{amsmath}
\usepackage{amsfonts}

% page numbers
\usepackage{fancyhdr}
\fancypagestyle{newstyle}{
\fancyhf{} % clear all header and footer fields
\fancyfoot[R]{\vspace{0.1in} \small \thepage}
\renewcommand{\headrulewidth}{0pt}
\renewcommand{\footrulewidth}{0pt}}
\pagestyle{newstyle}

% geometry of the page
\usepackage[top=1in, bottom=1in, left=1.625in, right=1.625in]{geometry}

% for enumerate* and itemize*
\usepackage{mdwlist}

% paragraph spacing
\setlength{\parindent}{0pt}
\setlength{\parskip}{2ex plus 0.4ex minus 0.2ex}

% useful packages
\usepackage{natbib}
\usepackage{epsfig}
\usepackage{url}
\usepackage{bm}


\begin{document}

\begin{center}
  \textbf{Week 5 Readings} \\
  C. Hidey \\
  \today
\end{center}

\paragraph{{\bf [EM1] Saif Mohammad. 2012}}
\text{} \\
Mohammad studied emotions in personal email and novels.  Through crowdsourcing, he created a large word-emotion lexicon from the Google n-grams corpus, for words appearing 120K times or more.  The word senses were assigned one of the 8 basic emotions.  This lexicon was then used to analyze workplace correspondence for gender differences and love/hate/suicide notes for emotions.  He also compared books across different genres.  What I liked about this paper was that it was explained well and the results were able to be visualized in word cloud form.  What I didn't like was that it required a heavy reliance on human judgments and mechanical turk.

\paragraph{{\bf [EM2] Rada Mihalcea and Carlo Strapparava. 2012}}
\text{} \\
Mihalcea and Stapparava looked at identifying emotions in songs.  They extracted various musical features from MIDI files that were matched up with lyrics.  They then annotated entire songs for six different emotions at the line level.  Using combined textual and musical features, they attempted to predict the emotions using linear regression.  What I liked about this paper was that it was novel work and is able to look at emotions in text in the context they should be in.  What I didn't like about the paper is that it would be difficult to replicate.  They used proprietary data that they paid for and would not be able to release.

\end{document}
