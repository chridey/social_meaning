\documentclass[12pt]{article}

% fonts
\usepackage[scaled=0.92]{helvet}   % set Helvetica as the sans-serif font
\renewcommand{\rmdefault}{ptm}     % set Times as the default text font

% dmb: not mandatory, but i recommend you use mtpro for math fonts.
% there is a free version called mtprolite.

% \usepackage[amssymbols,subscriptcorrection,slantedGreek,nofontinfo]{mtpro2}

\usepackage[T1]{fontenc}
\usepackage{amsmath}
\usepackage{amsfonts}

% page numbers
\usepackage{fancyhdr}
\fancypagestyle{newstyle}{
\fancyhf{} % clear all header and footer fields
\fancyfoot[R]{\vspace{0.1in} \small \thepage}
\renewcommand{\headrulewidth}{0pt}
\renewcommand{\footrulewidth}{0pt}}
\pagestyle{newstyle}

% geometry of the page
\usepackage[top=1in, bottom=1in, left=1.625in, right=1.625in]{geometry}

% for enumerate* and itemize*
\usepackage{mdwlist}

% paragraph spacing
\setlength{\parindent}{0pt}
\setlength{\parskip}{2ex plus 0.4ex minus 0.2ex}

% useful packages
\usepackage{natbib}
\usepackage{epsfig}
\usepackage{url}
\usepackage{bm}


\begin{document}

\begin{center}
  \textbf{Week 6 Readings} \\
  C. Hidey \\
  \today
\end{center}

\paragraph{{\bf [SI1]Riloff, Ellen, Qadir, Ashequl, Surve, Prafulla, Silva, Lalindra De, Gilbert, Nathan and Huang, Ruihong.}}
\text{} \\
Riloff et al. examined sarcasm in context.  They attempt to identify both a situation and sentiment, and consider that there may be sarcasm if they are opposite polarity.  They used a bootstrapping approach starting from positive and negative seed words to learn contexts.  What I liked about their paper is that it was easy to read and intuitive. What I didn't like about their approach was that there were was some manual involvement that may make it difficult to reproduce.

\paragraph{{\bf [SI2] Tepperman, Joseph, David Traum, and Shrikanth Narayanan}}
\text{} \\
Tepperman et al. examined the use of sarcasm in speech, focusing specifically on utterances containing the phrase ``yeah, right''.  They used objective cues such as laughter, 
pauses, and gender, among others.  They also used prosodic and spectral features derived from speech to improve their predictions.  What I liked about this paper was that their task was well-defined, intuitive, and the data backed up their hypothesis.  What I didn't like about this paper was that they intentionally limited themselves to a very specific linguistic situation that may not generalize to other data.

\end{document}
