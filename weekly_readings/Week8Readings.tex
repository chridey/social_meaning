\documentclass[12pt]{article}

\input{preamble}

\begin{document}

\begin{center}
  \textbf{Week 8 Readings} \\
  C. Hidey \\
  \today
\end{center}

\paragraph{{\bf [AD1] Murakami, Akiko and Rudy Raymond. 2010}}
\text{} \\
Murakami and Raymond model agreement and disagreement.  They examine a combination of text-based and link-based methods to create a classifier.  They frame the problem as a Max Cut problem and attempt to partition users into groups.  The maximization is done over the links between users and attempts to maximize the reaction coefficient between users in different groups.  I liked that this paper took a unique approach to agreement and disagreement, although I thought that the research was overly dependent on the hand-tuning of the parameters for the reaction coefficient.

\paragraph{{\bf [AD2] Bender, Emily M, Jonathan T Morgan, Meghan Oxley, Mark Zachry, Brian Hutchinson, Alex Marin, Bin Zhang, and Mari Ostendorf. 2011}}
\text{} \\
Bender et al. create a corpus for authority claims and agreement/disagreement in Wikipedia talk pages.  An authority claim is a reason given by the author for why their authority should be respected, and it can be a claim based on experience, education/training, an appeal to an authority or rules, or suggestions based on social norms.  The corpus is also annotated for alignment - when an author agrees or disagrees with another author - and they classified these alignments into explicit or implicit (criticism/sarcasm/praise/etc).  They also analyzed the corpus and found that unregistered users versus administrators are likely to make different authority claims.  I liked that this research considered the annotation of appeal to authority.  It is intuitive that most people would provide evidence of their authority but is not often considered in agreement tasks.  However, considering that the primary goal of this paper is to describe an annotation process, I think they could have been more explicit in how examples were annotated.

\end{document}
